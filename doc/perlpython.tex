\include{header}

\usepackage{gensymb}
\usepackage{hyphenat}
\usepackage{subfig}

%\setcounter{tocdepth}{3}
\begin{document}

\author{Hannes Röst}
\title{Executing Perl in Python}
\maketitle

On the Python wiki, there is some documentation on how to work with
other languages:
\url{http://wiki.python.org/moin/IntegratingPythonWithOtherLanguages}

\section{PyPerl}
I had success with PyPerl (Mandrive thing). I had to donwload the
original source code and then apply those 10 patches by hand. But at
least it worked, here is an example that creates a GENE object and
inserts it into the database (yes database adaptors work correctly).

\begin{verbatim}
import perl
perl.eval( "use lib './' ")
perl.require( 'DDB::GENE' )

obj = perl.callm("new", 'DDB::GENE'  )
obj[ '_ensembl_stable_id' ] = 9
obj[ '_experiment_key' ] = 42
obj.addignore_setid()
\end{verbatim}

%#perl.callm( 'addignore_setid', obj, myhash)
%myhash = perl.get_ref( '%' )
%h = {'ensembl_stable_id':  3}
%myhash.update( h )

%TODO: intercept method calls
Now we still have to create objects that we can call as we want. We dont
want the object to be a hash because later on we want to have attributes
and not elements of a dictionary. Here is a way to catch method /
attribute calls in python we could use:
\url{http://stackoverflow.com/questions/2704434/intercept-method-calls-in-python}

Here are more websites:
\url{http://search.cpan.org/~gaas/pyperl-1.0/perlmodule.pod}

\url{http://wiki.python.org/moin/PyPerl}



\newpage

Here I used the object abstraction using the Protein object that inherits from
the \url{PerlWrapper} object and has a \url{ddb_api} as a metaclass. I create
a wrapper object \url{Protein} that has all the same methods and attributes as
the Perl object and more can be added.

\begin{verbatim}
>>> p = Protein()
called init DDB::PROTEIN
try to set _file_key to
try to set _test to 0
try to set _parse_key to 0
try to set _experiment_key to
try to set _sequence_key to
try to set _protein_type to
try to set _comment to
try to set _gene_key to 0
try to set _probability to 0
try to set _mark_warning to
try to set _id to
>>> p.set_sequence_key( 42 )
try to set _sequence_key to 42
>>> p.set_experiment_key( 1 )
try to set _experiment_key to 1
>>> p.set_parse_key( 1 )
try to set _parse_key to 1
>>> p.set_probability( 0.6 )
try to set _probability to 0.6
>>> p.addignore_setid()
try to get addignore_setid
'0'
>>> p.get_sequence_key()
try to get _sequence_key
'42'
>>>
>>> sequence = p.get_sequence()
try to get get_sequence
>>> sequence.get_len()
7
>>> sequence.get_sequence()
'PROTEIN'
\end{verbatim}


\newpage
\section{Other Methods}
I found those but do not know whether they work\ldots

\url{http://www.boriel.com/2007/01/21/calling-perl-from-python/}

separate perl process to execute it (over STDIN):
\url{http://pypi.python.org/pypi/perlprocess/0.1}

execute through Perl interpreter:
\url{http://pypi.python.org/pypi/PythonPerl/0.9}

Commercial Bridgekepper converts Perl and Python source 
\url{http://www.crazy-compilers.com/bridgekeeper/}
\end{document}
