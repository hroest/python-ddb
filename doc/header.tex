% Autor Tobias Erbsland.
% Veröffentlicht unter The GNU Free Documentation License.
% modifiziert von Hannes Röst
% Version 1.2
%
% A. DOKUMENTKLASSE
% ---------------------------------------------------------------------------
%

%
%  1. Definieren der Dokumentklasse.
%     Wir verwenden die KOMA-Script Klasse 'scr...' 
%
\documentclass[
  pdftex,           %PDFTex verwenden da wir ausschliesslich ein PDF erzeugen.
  a4paper,          %Wir verwenden A4 Papier.
  oneside,          %Einseitiger Druck.
  12pt,             %12 Punkte, besser geeignet für A4.
  halfparskip,      %Halbe Zeile Abstand zwischen Absätzen.
  %chapterprefix,    %Kapitel mit 'Kapitel' anschreiben.
  %headsepline,      %Linie nach Kopfzeile.
  %footsepline,      %Linie vor Fusszeile.
  bibtotocnumbered, %Literaturverzeichnis im Inhaltsverz. nummeriert einfügen.
  idxtotoc          %Index ins Inhaltsverzeichnis einfügen.
]{scrartcl}


%
%  2. Festlegen der Zeichencodierung des Dokuments und des Zeichensatzes.
%     Wir verwenden 'UTF-8' für das Dokument,
%     und die 'T1' codierung für die Schrift.
%
\usepackage[T1]{fontenc}
\usepackage[utf8]{inputenc}

%  3. Packet für die Index-Erstellung laden.
%     makeindex ist ein Programm zur Erstellung von Stichwortverzeichnissen für LaTeX-Dokumente
%
\usepackage{makeidx}

%
%  4. Neue deutsche Rechtschreibung mit 'ngerman'. Es kommt es zu einer Übersetzung im Dokument
%     (Inhaltsverzeichnis statt table of contents, Daten). Außerdem wird nach deutscher
%     Rechtschreibung getrennt.
\usepackage[ngerman, english]{babel}


%
%  5. Paket für Anführungszeichen laden.
%     Wir setzen den Stil auf 'swiss', und verwenden so die Schweizer Anführungszeichen.
%
%\usepackage[style=swiss]{csquotes}


%
%  6. Paket für erweiterte Tabelleneigenschaften.
%
\usepackage{array}

%
%  7. Paket um Grafiken im Dokument einbetten zu können.
%     Im PDF sind GIF, PNG, und PDF Grafiken möglich.
%
\usepackage{graphicx}

%
%  8. Pakete für mathematischen Textsatz.
%     siehe dazu auch http://www.andy-roberts.net/misc/latex/latextutorial10.html
\usepackage{amsmath}
\usepackage{amssymb}
\usepackage{dsfont}
\usepackage{mathtools}

%
%  9. Paket um Textteile drehen zu können.
%     \begin{turn}{Winkel} Text \end{turn}
\usepackage{rotating}

%
% 10. Paket für Farben an verschieden Stellen. 
%     Wir definieren zusätzliche benannte Farben.
%     Mit dem Befehl \pagecolor{Farbe} wird die Hintergrundfarbe, und mit dem 
%     Befehl \color{Farbe} die Textfarbe bestimmt. 
%     http://www.willemer.de/informatik/text/texcolor.htm
\usepackage{color}

%
% 11. Paket für spezielle PDF features.
%
%     pdftitle     Titel des PDF Dokuments.
%     pdfauthor    Autor des PDF Dokuments.
%     pdfsubject   Thema des PDF Dokuments.
%     pdfcreator   Erzeuger des PDF Dokuments.
%     pdfkeywords  Schlüsselwörter für das PDF.
%     pdfpagemode  Inhaltsverzeichnis anzeigen beim Öffnen
%     pdfdisplaydoctitle     Dokumenttitel statt Dateiname anzeigen.
%     pdflang      Sprache des Dokuments.
\usepackage[
  pdfpagemode=UseOutlines,
  pdfdisplaydoctitle=true,
  %pdflang=en,
  pdfauthor={Hannes Roest},
  %pdftitle={Hannes Roest},
  colorlinks=true,
  linkcolor=black,
  citecolor=black,
  filecolor=black,
  pagecolor=black,
  urlcolor=black
%  frenchlinks=true
]{hyperref}

%
% 12. Font wenn die Schrift nicht schön aussieht, verwende eine andere
%     siehe auch http://www.matthiaspospiech.de/latex/vorlagen/allgemein/preambel/fonts/
%\usepackage{mathptmx}
%\usepackage[scaled=.90]{helvet}
%\usepackage{courier}


%
% 13. Line Spacing
%     This might give better (and might give worse) results

%\usepackage{setspace}
%\singlespacing        %% 1-spacing (default)
%\onehalfspacing       %% 1,5-spacing
%\doublespacing        %% 2-spacing


%
% 14. Kopf und Fusszeilen
%

\usepackage{fancyhdr}
\pagestyle{fancy}

\fancyhf{} % aktuelle header and footer löschen
% Buchstabencodes für \fancyhead und \fancyfoot sind:
% E     Even page
% O     Odd page
% L     Left
% C     Center
% R     Right
% Können gemixt werden, also auch: \fancyhead[RO,RE]{irgendetwas} für 
%          "rechts oben auf geraden und ungeraden seiten"
% "Gerade" und "ungerade" Seite werden unterschieden für Bücher, wo man die
% Seitenzahl immer aussen haben möchte.
%\renewcommand{\chaptermark}[1]{\markboth{#1}{}}
\renewcommand{\sectionmark}[1]{\markright{#1}{}}
\fancyhead[L]{\thechapter \phantom{L}  \leftmark}
\fancyhead[C]{}
\fancyhead[R]{\thesection \phantom{L} \rightmark}
\fancyfoot[L]{}
\fancyfoot[C]{\thepage }
%Linie vor/nach Fuss/Kopfzeile
\renewcommand{\headrulewidth}{0.0pt} 
\renewcommand{\footrulewidth}{0.0pt}
%\addtolength{\headheight}{0.5pt} % platz machen für die linie
%\addtolength{\footheight}{0.5pt} % platz machen für die linie
\fancypagestyle{plain}{%
       \fancyhead{} % kopfzeilen auf leeren seiten loswerden
       \renewcommand{\headrulewidth}{0pt} % ... und auch die linie
       \renewcommand{\footrulewidth}{0pt} % ... und auch die linie
}

%%
%% 15. Schöne Boxen
%%
%\usepackage{fancybox}

%% File Extensions of Graphics %%%%%%%%%%%%%%%%%%%%%%%%%%%%%%
%% ==> This enables you to omit the file extension of a graphic.
%% ==> "\includegraphics{title.eps}" becomes "\includegraphics{title}".
%% ==> If you create 2 graphics with same content (but different file types)
%% ==> "title.eps" and "title.pdf", only the file processable by
%% ==> your compiler will be used.
%% ==> pdfLaTeX uses "title.pdf". LaTeX uses "title.eps".
\DeclareGraphicsExtensions{.pdf,.jpg,.png}

%% Martina's preferences %%%%%%%%%%%%%%%%%%%%%%%%%%%%%%%%%%%%
%\setlength{\parskip}{6pt} %Legt den Abstand zwischen den nachfolgenden Absätzen fest. 
\usepackage{url} %besser mit hypperref
%\usepackage[paren, plain]{fancyref}
%\newcommand{\species}[1]{\textsl{#1}}  %%formating species names
%\setcounter{fignumdepth}{1} %%How can figurs be numbered independently of the chapter


%% Other Packages %%%%%%%%%%%%%%%%%%%%%%%%%%%%%%%%%%%%%%%%%%%
%\usepackage{a4wide} %%Smaller margins = more text per page.
%\usepackage{longtable} %%For tables, that exceed one page


%%Attention: the correct dash is the following: -
